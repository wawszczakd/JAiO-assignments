\documentclass[12pt]{article}
\usepackage[utf8]{inputenc}
\usepackage[T1]{fontenc}
\usepackage{indentfirst}
\usepackage{amsmath}
\usepackage{amssymb}
\usepackage{natbib}
\usepackage{graphicx}
\usepackage{float}
\usepackage[a4paper, margin = 2 cm]{geometry}
\usepackage{fancyhdr}
\usepackage{wrapfig}
\usepackage{hyperref}

\title{JAiO zadanie 3}
\author{Dominik Wawszczak}
\date{2023-05-25}

\begin{document}
	\setlength{\parindent}{0 cm}
	
	Dominik Wawszczak
	
	numer indeksu: 440014
	
	numer grupy: 6
	
	\bigskip
	\hrule
	\bigskip
	
	\textbf{Zadanie 3}
	
	\medskip
	
	Rozpocznijmy od udowodnienia, że język \(L\) nie jest obliczalny.
	
	\medskip
	
	Przypuśćmy nie wprost, że \(L = L \left( \mathcal{H} \right)\), dla pewnej
	maszyny Turinga \(\mathcal{H}\), która zawsze terminuje i zostawia na taśmie
	\(1\), jeżeli dostanie na wejściu zakodowaną parę \(\left\langle u_{1},
	u_{2} \right\rangle\), gdzie \(u_{1}\) i \(u_{2}\) są kodami podobnych
	maszyn Turinga, lub \(0\) w przeciwnym wypadku. Wskażemy redukcję
	\[ \text{HALT}_{\varepsilon} \ \leqslant_{f} \ L \text{,} \]
	co da oczywistą sprzeczność, gdyż problem \(\text{HALT}_{\varepsilon}\) jest
	nierozstrzygalny.
	
	\medskip
	
	Weźmy funkcję \(f\) o następującej definicji:
	\[ f \left( \text{kod} \left( \mathcal{M} \right) \right) \ = \ \left\langle
	\text{kod} \left( \mathcal{M} \right), \text{kod} \left( \mathcal{M}'
	\right) \right\rangle \text{,} \]
	gdzie \(\mathcal{M}\) jest dowolną maszyną Turinga, a \(\mathcal{M}'\)
	maszyną Turinga działającą w sposób opisany poniżej:
	\begin{itemize}
		\item jeśli wejściowym napisem jest \(\varepsilon\), to maszyna
		      \(\mathcal{M}'\) najpierw symuluje działanie maszyny
		      \(\mathcal{M}\) na tym napisie, a następnie w miejsce wystąpienia
		      pierwszego blanka wpisuje \(1\) i kończy działanie;
		\item w przeciwnym wypadku symuluje działanie maszyny \(\mathcal{M}\) na
		      napisie wejściowym, po czym kończy działanie.
	\end{itemize}
	Funkcja \(f\) jest oczywiście obliczalna.
	
	\medskip
	
	Zauważmy, że
	\[ \mathcal{M} \ \text{terminuje na} \ \varepsilon \quad \iff \quad
	\text{maszyny} \ \mathcal{M} \ \text{oraz} \ \mathcal{M}' \ \text{nie są
	podobne,} \]
	czyli równoważnie
	\[ \text{kod} \left( \mathcal{M} \right) \ \in \ \text{HALT}_{\varepsilon}
	\quad \iff \quad \left\langle \text{kod} \left( \mathcal{M} \right),
	\text{kod} \left( \mathcal{M}' \right) \right\rangle \ \notin \ L
	\text{.} \]
	Wynika to z tego, że jeżeli \(\mathcal{M}\) terminuje na \(\varepsilon\),
	to \(\mathcal{M}'\) również terminuje na \(\varepsilon\), jednak daje inny
	wynik, więc nie są podobne. Z drugiej strony, jeśli maszyny \(\mathcal{M}\)
	oraz \(\mathcal{M}'\) nie są podobne, to \(\mathcal{M}\) terminuje na
	\(\varepsilon\), ponieważ w przeciwnym wypadku, obie maszyny \(\mathcal{M}\)
	oraz \(\mathcal{M}'\) nie terminowałyby na \(\varepsilon\), a na wszystkich
	innych wejściach dawałyby ten sam wynik lub obie by nie terminowały, zatem
	byłyby podobne. Z implikacji w obie strony dostajemy więc równoważność.
	
	\medskip
	
	Z powyższego wynika, że język \(L\) nie jest obliczalny.
	
	\bigskip
	
	Udowodnimy teraz, że dopełnienie języka \(L\) jest częściowo obliczalne.
	
	\medskip
	
	Stworzymy maszynę Turinga \(\mathcal{M}\), która na wejściu będzie
	przyjmować zakodowaną parę \(\left\langle u_{1}, u_{2} \right\rangle\) i
	terminować, jeżeli \(u_{1}\) i \(u_{2}\) są kodami maszyn Turinga, które nie
	są podobne. Maszyna \(\mathcal{M}\) będzie symulować działanie maszyn
	\(\mathcal{M}_{u_{1}}\) i \(\mathcal{M}_{u_{2}}\) na wszystkich możliwych
	wejściach \(w \in \left\{ 0, 1 \right\} ^ {\ast}\), aż znajdzie wejście na
	którym dają one różne wyniki, stwierdzając przy tym, że nie są podobne.
	
	\medskip
	
	Biorąc dowolną obliczalną bijekcję \(g : \mathbb{Z}^{+} \cup \left\{ 0
	\right\} \rightarrow \left\{ 0, 1 \right\} ^ {\ast}\) możemy łatwo symulować
	działanie maszyn \(\mathcal{M}_{u_{1}}\) i \(\mathcal{M}_{u_{2}}\) na
	wszystkich możliwych wejściach, obliczająć wartości funkcji \(g\) dla
	kolejnych liczb całkowitych nieujemnych. Przykładem takiej bijekcji jest
	funkcja:
	\[ g \left( n \right) \ = \ \text{zapis binarny liczby} \ n + 1 \ \text{bez
	pierwszego znaku.} \]
	Wówczas kilka pierwszych wartości funkcji \(g\) to: \(\varepsilon, 0, 1, 00,
	01, 10, 11, 000, 001, 010, 011, 100 \ldots\).
	
	\medskip
	
	Problem napotkamy w momencie, gdy co najmniej jedna z maszyn
	\(\mathcal{M}_{u_{1}}\), \(\mathcal{M}_{u_{2}}\) nie terminuje na pewnym
	wejściu, a obie terminują i dają inne wyniki na innym, jeszcze nie
	rozpatrzonym wejściu. Aby rozwiązać ten problem nie będziemy symulować
	maszyn na wszystkich możliwych wejściach po kolei, lecz sumulację będziemy
	przeprowadzać równolegle, na coraz większej liczbie wejść.
	
	\medskip
	
	Na ćwiczeniach dowodzone było, że dowolna maszyna Turinga, zamiast działać
	na taśmie jednowymiarowej, może działać na taśmie dwuwymiarowej. Główna idea
	opierała się na tym, żeby na taśmie trzymać wszystkie trójki \(\left(
	\text{wiersz}, \text{kolumna}, \text{wartość} \right)\) wykorzystywanych
	pól, jednak nie będziemy zagłębiać się w szczegóły.
	
	\medskip
	
	Maszyna \(\mathcal{M}\) będzie działać w fazach, trzymając numer aktualnej
	fazy na taśmie. W \(n\)-tej fazie będzie przeprowadzać po jednym kroku
	symulacji obu maszyn \(\mathcal{M}_{u_{1}}\) i \(\mathcal{M}_{u_{2}}\)
	na pierwszych \(n\) wejściach, tj. \(\left\{ g \left( 0 \right), g \left( 1
	\right), g \left( 2 \right), \ldots, g \left( n - 1 \right) \right\}\).
	
	\medskip
	
	Kilka początkowych wierszy (taśm) naszej dwuwymiarowej taśmy jest potrzebne
	na istotne rzeczy takie jak np.: kody \(u_{1}\) i \(u_{2}\), numer aktualnej
	fazy i inne informacje niezbędne do pełnego sformalizowania działania
	maszyny \(\mathcal{M}\). Niech więc \(k\) bedzie numerem pierwszego
	nieużywanego wiersza. Wykorzystamy go do symulowania działania maszyny
	\(\mathcal{M}_{u_{1}}\) na \(g \left( 0 \right)\). Wiersz \(k + 1\)
	natomiast zostanie wykorzystany do symulacji maszyny \(\mathcal{M}_{u_{2}}\)
	na \(g \left( 0 \right)\). Kolejne wiersze będą wykorzystywane do
	symulowania \(\mathcal{M}_{u_{1}}\) albo \(\mathcal{M}_{u_{2}}\), w
	zależności od parzystości numeru wiersza, na odpowiednich wartościach
	funkcji \(g\).
	
	\medskip
	
	Jeśli maszyny \(\mathcal{M}_{u_{1}}\) i \(\mathcal{M}_{u_{2}}\) nie są
	podobne, to maszyna \(\mathcal{M}\) w skończonym czasie znajdzie wejście,
	na którym obie maszyny \(\mathcal{M}_{u_{1}}\), \(\mathcal{M}_{u_{2}}\)
	terminują, dając różne wyniki, po czym sama również sterminuje.
	
	\bigskip
	
	Aby dokończyć rozwiązanie zadania, należy zauważyć, że język \(L\) nie jest
	częściowo obliczalny, gdyż w przeciwnym wypadku język \(L\) byłby
	obliczalny, co było udowodnione na wykładzie.
	
	\bigskip
	
	Z powyższego wynika, że odpowiedzi na pytania zawarte w podpunktach a), b)
	oraz c) to kolejno ,,Nie.'', ,,Nie.'' oraz ,,Tak.'', co kończy rozwiązanie
	zadania.
	\begin{flushright}
		\(\Box\)
	\end{flushright}
\end{document}
