\documentclass[12pt]{article}
\usepackage[utf8]{inputenc}
\usepackage[T1]{fontenc}
\usepackage{indentfirst}
\usepackage{amsmath}
\usepackage{amssymb}
\usepackage{natbib}
\usepackage{graphicx}
\usepackage{float}
\usepackage[a4paper, margin = 2 cm]{geometry}
\usepackage{fancyhdr}
\usepackage{wrapfig}
\usepackage{hyperref}

\title{JAiO zadanie 4}
\author{Dominik Wawszczak}
\date{2023-06-06}

\begin{document}
	\setlength{\parindent}{0 cm}
	
	Dominik Wawszczak
	
	numer indeksu: 440014
	
	numer grupy: 6
	
	\bigskip
	\hrule
	\bigskip
	
	\textbf{Zadanie 4}
	
	\medskip
	
	Rozpocznijmy od udowodnienia, że dany w treści zadania problem jest w klasie
	NP. Łatwo wskazać maszynę Turinga, która dla danej formuły \(\varphi\)
	niedeterministycznie wybiera wartościowanie \(v\), a następnie oblicza
	\(\text{score} \left( \varphi, v \right)\) i sprawdza czy spełnione jest \(3
	\leqslant \text{score} \left( \varphi, v \right) \leqslant 5\). Oczywiście
	wszystkie opisane wyżej kroki można przeprowadzić w czasie wielomianowym.
	
	\medskip
	
	Pozostaje pokazać, że problem ten jest NP-trudny. W tym celu wskażemy
	wielomianową redukcję problemu 3-CNF-SAT do problemu z treści zadania. Weźmy
	więc dowolną formułę \(\psi\) w postaci 3-CNF. Stwórzmy nową formułę
	\(\varphi\) w następujący sposób:
	\begin{itemize}
		\item do każdej klauzuli w \(\psi\) dodajemy dwa razy literał \(x\);
		\item do formuły \(\psi\) dodajemy klauzulę \(\left( x \vee x \vee x
		      \vee x \vee x \right)\).
	\end{itemize}
	Oczywiście mając formułę \(\psi\), formułę \(\varphi\) można łatwo obliczyć
	w czasie wielomianowym. Zakładamy przy tym, że zmienna \(x\) jest nową
	zmienną, niewystępującą w formule \(\psi\).
	
	\medskip
	
	Przykładowo, dla
	\[ \psi \ = \ \left( \neg x_{1} \vee x_{2} \right) \wedge \left( x_{1} \vee
	\neg x_{2} \vee \neg x_{3} \right) \wedge \left( x_{3} \right) \]
	będziemy mieli
	\[ \varphi \ = \ \left( \neg x_{1} \vee x_{2} \vee x \vee x \right) \wedge
	\left( x_{1} \vee \neg x_{2} \vee \neg x_{3} \vee x \vee x \right) \wedge
	\left( x_{3} \vee x \vee x \right) \wedge \left( x \vee x \vee x \vee x \vee
	x \right) \text{.} \]
	
	\medskip
	
	Udowodnimy teraz następującą równoważność:
	\[ \text{formuła} \ \psi \ \text{jest spełnialna} \quad \iff \quad
	\text{istnieje wartościowanie} \ v \ \text{takie, że} \ 3 \leqslant
	\text{score} \left( \varphi, v \right) \leqslant 5 \text{.} \]
	
	\medskip
	
	Załóżmy, że formuła \(\psi\) jest spełniona przez wartościowanie \(u\).
	Wówczas \(1 \leqslant \text{score} \left( \psi, u \right)\). Ponadto
	\(\text{score} \left( \psi, u \right) \leqslant 3\), gdyż formuła \(\psi\)
	jest w formacie 3-CNF. Niech wartościowanie \(v\) będzie takie samo jak
	\(u\) z dodatkowym przypisaniem \(x = \top\). Wtedy w każdej klauzuli
	formuły \(\varphi\) są co najmniej \(3\) literały będące prawdziwe przy
	wartościowaniu \(v\) -- co najmniej jeden z wartościowania \(u\) i dwa razy
	\(x\), a w ostatniej klauzuli \(5\) razy \(x\). Zachodzi więc \(3 \leqslant
	\text{score} \left( \varphi, v \right)\). Dodatkowo \(\text{score} \left(
	\varphi, v \right) \leqslant 5\), ponieważ w każdej klauzuli jest co
	najwyżej \(5\) literałów.
	
	\medskip
	
	Załóżmy teraz, że dla pewnego wartościowania \(v\) spełnione jest \(3
	\leqslant \text{score} \left( \varphi, v \right) \leqslant 5\). Łatwo
	zauważyć, że w wartościowaniu tym musi być \(x = \top\), gdyż w przeciwnym
	wypadku ostatnia klauzula formuły \(\varphi\) byłaby fałszywa. Skoro \(3
	\leqslant \text{score} \left( \varphi, v \right)\), to w każdej klauzuli
	formuły \(\varphi\) jest co najmniej jeden literał inny niż \(x\) będący
	prawdziwy przy wartościowaniu \(v\), ponieważ każda klauzula zawiera
	dokładnie dwa literały \(x\) (z wyjątkiem ostatniej klauzuli). Oznacza to,
	że wartościowanie \(u\) powstałe poprzez usunięcie z wartościowania \(v\)
	przypisania \(x = \top\) spełnia formułę \(\psi\).
	
	\medskip
	
	Powyższa redukcja implikuje, że problem opisany w treści zadania jest
	NP-zupełny, co kończy rozwiązanie zadania.
	\begin{flushright}
		\(\Box\)
	\end{flushright}
\end{document}
