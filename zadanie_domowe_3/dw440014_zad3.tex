\documentclass[12pt]{article}
\usepackage[utf8]{inputenc}
\usepackage[T1]{fontenc}
\usepackage{indentfirst}
\usepackage{amsmath}
\usepackage{amssymb}
\usepackage{natbib}
\usepackage{graphicx}
\usepackage{float}
\usepackage[a4paper, margin = 2 cm]{geometry}
\usepackage{fancyhdr}
\usepackage{wrapfig}
\usepackage{hyperref}

\title{JAiO zadanie 3}
\author{Dominik Wawszczak}
\date{2023-05-??}

\begin{document}
	\setlength{\parindent}{0 cm}
	
	Dominik Wawszczak
	
	numer indeksu: 440014
	
	numer grupy: 6
	
	\bigskip
	\hrule
	\bigskip
	
	\textbf{Zadanie 3}
	
	\medskip
	
	Rozpocznijmy od udowodnienia, że język \(L\) nie jest obliczalny.
	
	\medskip
	
	Przypuśćmy nie wprost, że istnieje maszyna turinga \(\mathcal{H}\), która
	zawsze terminuje i zostawia na taśmie \(1\), jeżeli wejściowy napis jest
	postaci \(u_1 \$ u_2\), gdzie \(u_1\) i \(u_2\) są kodami podobnych maszyn
	Turinga, lub \(0\) w przeciwnym wypadku. Rozwiążemy problem
	\(HALT_{\varepsilon}\) wykorzystując maszynę \(\mathcal{H}\), co da
	oczywistą sprzeczność, gdyż problem \(HALT_{\varepsilon}\) jest
	nierozstrzygalny.
	
	\medskip
	
	Niech więc \(\mathcal{M}\) będzie dowolną maszyną Turinga. Stwórzmy maszynę
	\(\mathcal{M}'\) działającą w następujący sposób:
	\begin{itemize}
		\item jeśli wejściowym napisem jest \(\varepsilon\), to maszyna
		      \(\mathcal{M}'\) najpierw symuluje działanie maszyny
		      \(\mathcal{M}\) na tym napisie, a następnie w miejsce wystąpienia
		      pierwszego blanka wpisuje \(1\) i kończy działanie;
		\item w przeciwnym wypadku symuluje działanie maszyny \(\mathcal{M}\) na
		      napisie wejściowym, a następnie kończy działanie.
	\end{itemize}
	Wykorzysując maszynę \(\mathcal{H}\) możemy stwierdzić czy maszyny
	\(\mathcal{M}\) i \(\mathcal{M}'\) są podobne.
	
	\medskip
	
	Zauważmy, że
	\[ \mathcal{M} \ \text{terminuje na} \ \varepsilon \quad \iff \quad
	\text{maszyny} \ \mathcal{M} \ \text{oraz} \ \mathcal{M}' \ \text{nie są
	podobne.} \]
	Wynika to z tego, że jeśli \(\mathcal{M}\) terminuje na \(\varepsilon\),
	to \(\mathcal{M}'\) również terminuje na \(\varepsilon\), jednak daje inny
	wynik. Z drugiej strony, jeśli maszyny \(\mathcal{M}\) oraz
	\(\mathcal{M}'\) nie są podobne, to \(\mathcal{M}\) nie terminuje na
	\(\varepsilon\), ponieważ w przeciwnym wypadku \(\mathcal{M}'\) również
	terminowałaby na \(\varepsilon\), dając jednak inny wynik. Z implikacji w
	obie strony dostajemy więc równoważność.
	
	\medskip
	
	Z powyższego wynika, że język \(L\) nie jest obliczalny.
\end{document}
