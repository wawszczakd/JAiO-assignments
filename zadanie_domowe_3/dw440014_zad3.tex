\documentclass[12pt]{article}
\usepackage[utf8]{inputenc}
\usepackage[T1]{fontenc}
\usepackage{indentfirst}
\usepackage{amsmath}
\usepackage{amssymb}
\usepackage{natbib}
\usepackage{graphicx}
\usepackage{float}
\usepackage[a4paper, margin = 2 cm]{geometry}
\usepackage{fancyhdr}
\usepackage{wrapfig}
\usepackage{hyperref}

\title{JAiO zadanie 3}
\author{Dominik Wawszczak}
\date{2023-05-25}

\begin{document}
	\setlength{\parindent}{0 cm}
	
	Dominik Wawszczak
	
	numer indeksu: 440014
	
	numer grupy: 6
	
	\bigskip
	\hrule
	\bigskip
	
	\textbf{Zadanie 3}
	
	\medskip
	
	Rozpocznijmy od udowodnienia, że język \(L\) nie jest obliczalny.
	
	\medskip
	
	Przypuśćmy nie wprost, że \(L = L \left( \mathcal{H} \right)\), dla pewnej
	maszyny Turinga \(\mathcal{H}\), która zawsze terminuje i zostawia na taśmie
	\(1\), jeżeli wejściowy napis jest postaci \(u_1 \$ u_2\), gdzie \(u_1\) i
	\(u_2\) są kodami podobnych maszyn Turinga, lub \(0\) w przeciwnym wypadku.
	Wskażemy redukcję
	\[ \text{HALT}_{\varepsilon} \ \leqslant_{f} \ L \text{,} \]
	co da oczywistą sprzeczność, gdyż problem \(\text{HALT}_{\varepsilon}\) jest
	nierozstrzygalny.
	
	\medskip
	
	Weźmy funkcję \(f\) o następującej definicji:
	\[ f \left( \text{kod} \left( \mathcal{M} \right) \right) \ = \ \left\langle
	\text{kod} \left( \mathcal{M} \right), \text{kod} \left( \mathcal{M}'
	\right) \right\rangle \text{,} \]
	gdzie \(\mathcal{M}\) jest dowolną maszyną Turinga, a \(\mathcal{M}'\)
	maszyną Turinga działającą w sposób opisany poniżej:
	\begin{itemize}
		\item jeśli wejściowym napisem jest \(\varepsilon\), to maszyna
		      \(\mathcal{M}'\) najpierw symuluje działanie maszyny
		      \(\mathcal{M}\) na tym napisie, a następnie w miejsce wystąpienia
		      pierwszego blanka wpisuje \(1\) i kończy działanie;
		\item w przeciwnym wypadku symuluje działanie maszyny \(\mathcal{M}\) na
		      napisie wejściowym, po czym kończy działanie.
	\end{itemize}
	Funkcja \(f\) jest oczywiście obliczalna.
	
	\medskip
	
	Zauważmy, że
	\[ \mathcal{M} \ \text{terminuje na} \ \varepsilon \quad \iff \quad
	\text{maszyny} \ \mathcal{M} \ \text{oraz} \ \mathcal{M}' \ \text{nie są
	podobne,} \]
	czyli równoważnie
	\[ \text{kod} \left( \mathcal{M} \right) \ \in \ \text{HALT}_{\varepsilon}
	\quad \iff \quad \left\langle \text{kod} \left( \mathcal{M} \right),
	\text{kod} \left( \mathcal{M}' \right) \right\rangle \ \notin \ L
	\text{.} \]
	Wynika to z tego, że jeżeli \(\mathcal{M}\) terminuje na \(\varepsilon\),
	to \(\mathcal{M}'\) również terminuje na \(\varepsilon\), jednak daje inny
	wynik. Z drugiej strony, jeśli maszyny \(\mathcal{M}\) oraz \(\mathcal{M}'\)
	nie są podobne, to \(\mathcal{M}\) terminuje na \(\varepsilon\), ponieważ w
	przeciwnym wypadku, obie maszyny \(\mathcal{M}\) oraz\(\mathcal{M}'\) nie
	terminowałaby na \(\varepsilon\), dając jednak inny wynik. Z implikacji w
	obie strony dostajemy więc równoważność.
	
	\medskip
	
	Z powyższego wynika, że język \(L\) nie jest obliczalny.
	
	\bigskip
	
	
\end{document}
