\documentclass[12pt]{article}
\usepackage[utf8]{inputenc}
\usepackage[T1]{fontenc}
\usepackage{indentfirst}
\usepackage{amsmath}
\usepackage{amssymb}
\usepackage{natbib}
\usepackage{graphicx}
\usepackage{float}
\usepackage[a4paper, margin = 2 cm]{geometry}
\usepackage{fancyhdr}
\usepackage{wrapfig}
\usepackage{hyperref}

\title{JAiO zadanie 2}
\author{Dominik Wawszczak}
\date{2023-04-20}

\begin{document}
	\setlength{\parindent}{0 cm}
	
	Dominik Wawszczak
	
	numer indeksu: 440014
	
	numer grupy: 6
	
	\bigskip
	\hrule
	\bigskip
	
	\textbf{Zadanie 2.1}
	
	\medskip
	
	Rozpatrzmy gramatykę bezkontekstową \(\mathcal{G} = \left( \{a, b\}, \{S, X,
	Y, Z\}, P, S \right)\) z produkcjami:
	\begin{itemize}
		\item \(S \longrightarrow aXbYa\),
		\item \(X \longrightarrow ZaXb \mid \varepsilon\),
		\item \(Y \longrightarrow bYaZ \mid \varepsilon\),
		\item \(Z \longrightarrow bZ \mid \varepsilon\).
	\end{itemize}
	
	\medskip
	
	Rozpocznijmy od udowodnienia, że \(X\) generuje wszystkie słowa postaci
	\(b^{n_{1}} a b^{n_{2}} a \ldots a b^{n_{k}}\), gdzie \(n_{k} = k - 1\). W
	tym celu skorzystamy z indukcji matematycznej po długości słowa \(w\)
	generowanego przez \(X\).
	
	\medskip
	
	\underline{Pierwszy krok}: Jeżeli \(\left| w \right| = 0\), to \(w =
	\varepsilon\), czyli \(k = 1\), a \(n_{1} = 0 = k - 1\).
	
	\medskip
	
	\underline{Krok indukcyjny}: Dla słowa \(w\) o długości \(n\) większej niż
	\(0\) pierwszą użytą produkcją musi być \(X \longrightarrow ZaXb\). Łatwo
	zauważyć, że \(Z\) generuje język \(b^{\ast}\), zatem dla pewnego \(w' =
	b^{n_{1}} a b^{n_{2}} a \ldots a b^{n_{k}}\), gdzie \(n_k = k - 1\),
	zachodzi
	\[ w \ = \ b^{n_{0}} a w' b \ = \ b^{n_{0}} a b^{n_{1}} a b^{n_{2}} a \ldots
	a b^{n_{k} + 1} \text{.} \]
	Zmieniając nieco oznaczenia kolejnych wykładników dostajemy tezę.
	
	\medskip
	
	Wykażemy teraz, że każde słowo postaci \(b^{n_{1}} a b^{n_{2}} a \ldots a
	b^{n_{k}}\), gdzie \(n_{k} = k - 1\) jest generowane przez \(X\). Niech więc
	\(w = b^{n_{1}} a b^{n_{2}} a \ldots a b^{n_{k}}\), gdzie \(n_{k} = k - 1\).
	Skorzystajmy \(k - 1\) razy z produkcji \(X \longrightarrow ZaXb\), a
	następnie z produkcji \(X \longrightarrow \varepsilon\), otrzymując
	\[ \underbrace{Z a Z a \ldots a Z a}_{k - 1 \ \text{wystąpień} \ Z a}
	b^{k - 1} \text{.} \]
	Z każdego nieterminala \(Z\) możemy uzyskać odpowiednią liczbę wystąpień
	smymbolu \(b\), otrzymując słowo \(w\), co kończy dowód.
	
	\medskip
	
	Analogicznie dowodzimy, że \(Y\) generuje wszystkie słowa postaci
	\(b^{m_{1}} a b^{m_{2}} a \ldots a b^{m_{l}}\), gdzie \(m_{1} = l - 1\).
	
	\medskip
	
	Z powyższego wnioskujemy, że \(S\), oprócz słowa \(a\), generuje słowa
	postaci
	\[ a b^{n_{1}} a b^{n_{2}} a \ldots a b^{n_{k}} b b^{m_{1}} a b^{m_{2}} a
	\ldots a b^{m_{l}} a \text{,} \quad \text{gdzie} \quad n_{k} = k - 1 \quad
	\text{oraz} \quad m_{1} = l - 1 \text{.} \]
	Inaczej, po zmienieniu oznaczeń i uproszczeniu, są to słowa postaci
	\[ a b^{n_{1}} a b^{n_{2}}a \ldots a b^{n_{k}} a \text{,} \quad \text{gdzie}
	\quad \underset{i \in \left[ 1, k \right] \cap \mathbb{Z}}{\exists} \ n_{i}
	= k \text{.} \]
	Łatwo też pokazać, że każde słowo tej postacji jest generowane przez \(S\).
	Dla \(w = a b^{n_{1}} a \ldots a b^{n_{k}} a\) takiego, że \(n_{i} = k\),
	dla konkretnego \(i \in \left[ 1, k \right] \cap \mathbb{Z}\), po
	skorzystaniu z produkcji \(S \longrightarrow aXbYa\) wystarczy z \(X\)
	wygenerować słowo \(b^{n_{1}} a \ldots a b^{n_{i - 1}} a b^{i - 1}\), a z
	\(Y\) słowo \(b^{k - i} a b^{n_{i + 1}} a \ldots a b^{n_{k}}\).
	
	\medskip
	
	Z powyższego wynika, że \(L \left( \mathcal{G} \right) = L_{\exists}\), co
	kończy rozwiązanie zadania.
	\begin{flushright}
		\(\Box\)
	\end{flushright}
	
	\newpage
	
	\textbf{Zadanie 2.2}
	
	\medskip
	
	Na początek zauważmy, że
	\[ L_{\forall} \ = \ \left\{ a \left( b^{n} a \right) ^ {n} \ : \ n \in
	\mathbb{Z}^{+} \cup \{0\} \right\} \text{,} \]
	co wynika wprost z definicji języka \(L_{\forall}\). Wnioskujemy stąd, że
	dla każdego \(n \in \mathbb{Z}^{+}\) istnieje dokładnie jedno słowo \(w \in
	L_{\forall}\) spełniające \(\#_{a} \left( w \right) = n\) i jest to słowo
	\(a \left( b^{n - 1} a \right) ^ {n - 1}\).
	
	\medskip
	
	Udowodnimy, że język \(L_{\forall}\) nie jest bezkontekstowy. W tym celu
	skorzystamy z lematu o pompowaniu dla języków bezkontekstowych. Niech
	więc \(N\) będzie stałą z lematu o pompowaniu dla języka \(L_{\forall}\).
	
	\medskip
	
	Weźmy słowo \(w = a \left( b^{N} a \right) ^ {N} \in L_{\forall}\). Wówczas
	\(|w| \geqslant N\), zatem istnieje faktoryzacja
	\[ w \ = \ prefix \cdot left \cdot infix \cdot right \cdot suffix \]
	taka, że dla dowolnego \(k \in \mathbb{Z}^{+} \cup \{0\}\) zachodzi
	\[ prefix \cdot left^{k} \cdot infix \cdot right^{k} \cdot suffix \ \in \
	L_{\forall} \text{,} \]
	przy czym \(\left| left \cdot right \right| \geqslant 1\) oraz \(\left| left
	\cdot infix \cdot right \right| \leqslant N\). Niech \(m = \left| left \cdot
	right \right|\). Oczywiście \(1 \leqslant m \leqslant N\).
	
	\medskip
	
	Z powyższego wnioskujemy, że
	\(\#_{a} \left( left \cdot infix \cdot right \right) \leqslant 1\), mamy
	więc dwa przypadki:
	\begin{enumerate}
		\item \(\#_{a} \left( left \cdot right \right) = 0\)
		      
		      Wtedy \(\#_{b} \left( left \cdot right \right) = \left| left \cdot
		      right \right| = m\). Rozpatrzmy słowo
		      \[ w' \ = \ prefix \cdot left^{2} \cdot infix \cdot right^{2}
		      \cdot suffix \text{.} \]
		      Z lematu o pompowaniu wynika, że \(w' \in L_{\forall}\). Zauważmy,
		      że
		      \[ \#_{a} \left( w' \right) \ = \ N + 1 \ = \ \#_{a} \left( w
		      \right) \quad \text{oraz} \quad \#_{b} \left( w' \right) \ = \
		      N^{2} + m \ > \ N^{2} \ = \ \#_{b} \left( w \right) \text{,} \]
		      zatem \(w' \neq w\). Otrzymujemy więc sprzeczność z poczynioną
		      na początku obserwacją.
		
		\item \(\#_{a} \left( left \cdot right \right) = 1\)
		      
		      Wówczas \(\#_{b} \left( left \cdot right \right) = \left| left
		      \cdot right \right| - 1 = m - 1\). Ponownie będziemy rozpatrywać
		      słowo
		      \[ w' \ = \ prefix \cdot left^{2} \cdot infix \cdot right^{2}
		      \cdot suffix \text{,} \]
		      które oczywiście należy do języka \(L_{\forall}\). Niech \(v = a
		      \left( b^{N + 1} a \right) ^ {N + 1} \in L_{\forall}\). Łatwo
		      zauważyć, że
		      \[ \#_{a} \left( w' \right) \ = \ \#_{a} \left( w \right) + 1 \ =
		      \ N + 2 \ = \ \#_{a} \left( v \right) \text{.} \]
		      Ponadto
		      \[ \#_{b} \left( w' \right) \ = \ \#_{b} \left( w \right) + m - 1
		      \ = \ N^{2} + m - 1 \ \leqslant \ N^2 + N - 1 \ < \ \left( N + 1
		      \right) ^ {2} \ = \ \#_{b} \left( v \right) \text{,} \]
		      toteż \(w' \neq v\). Tym razem również dostajemy sprzeczność z
		      obserwacją z początku, co kończy rozwiązanie zadania.
	\end{enumerate}
	\begin{flushright}
		\(\Box\)
	\end{flushright}
\end{document}
